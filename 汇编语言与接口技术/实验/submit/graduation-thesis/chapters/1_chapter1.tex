
\chapter{实验目的}
掌握 Windows 系统汇编程序的基本结构,掌握基本的汇编指令,用汇编实现程序的分支和循环等结构。熟练使用 Visual Studio 进行汇编程序
的编写与调试;深入理解汇编数据类型之间的差异,学习调用 C 语言函数,综合提高汇编编程能力。
同时从汇编的角度加强对C语言优化的理解。

\chapter{实验环境}

\begin{table}[htbp]
    \linespread{2}
    \zihao{3}
    \centering
    \setlength{\tabcolsep}{10mm}%7可随机设置,调整到适合自己的大小为止
    \caption{环境信息}\label{环境信息}
    \begin{tabular}{cc}
        \hline
        名称     & 详情                       \\ \hline
        CPU      & Intel(R) Core(TM) i5-8265U \\
        操作系统 & Windows 10 家庭中文版      \\
        IDE      & Visual Studio 2019         \\
        汇编器   & MASM32 V11                 \\ \hline
    \end{tabular}
\end{table}

\chapter{实验三——大数相乘}

\section{实验目的}
% 这里插入一个参考文献,仅作参考
要求实现两个十进制大整数的相乘(100 位以上),输出乘法运算的结果。

\section{实验设计}

\subsection{大数存储形式}
采用DWORD数组以小端的形式存放大数,并在计算中可以令数组中的元素暂时大于10,
在最后再统一进位,这样处理的优点如下:
\begin{itemize}
    \item 采用小端储存,可以直接从最低位进行逐位运算,不用逆序遍历
    \item 采用小端储存,不用考虑越界的问题
    \item 使用DWORD储存,运算中的中间值可以很大,最后统一进位,不用在相乘过程中频繁进位
\end{itemize}

\subsection{程序流程}
程序流程如下:
\begin{enumerate}
    \item scanf读入两个大整数,以字符串的形式存放
    \item 调用strlen得到两个字符串的长度
    \item 将字符串形式的整数转化为数组形式
    \item 将两个数组逐位相乘并加到结果数组中
    \item 对结果数组进行进位处理,使每一个元素的大小属于[0, 9]
    \item 得到结果数组的长度
    \item 根据负号的个数判断是否需要输出’-‘
    \item 输出结果
\end{enumerate}


\section{具体实现}
该节介绍了代码中的部分核心算法和具体代码。

\subsection{字符串转化为逆序数组}
\begin{algorithm}[H]
    \SetAlgoLined
    \KwIn{存储字符串的变量首地址、字符串的长度}
    \KwResult{数组的首地址}
    $idx \leftarrow 0$ \;
    \While{idx < len}{
        \If {string[len - 1 - idx] == ’-’} {
            negNum += 1 \;
        }
        numbers[idx] = string[len - 1 - idx] - '0'\;
        idx++;
    }

    \caption{字符转无符号整数并逆序存储}
\end{algorithm}

即倒序遍历字符串,把每个字符减去'0'后便得到数字,并存在数组的相应位置,
具体汇编代码如下,需要注意的是,字符减去'0'之后是一个8位的整数,如果要
把这个数字放入DWORD数组中,则操作是先把这个八位的数字放入AL寄存器中,再
把EAX寄存器中的值赋给DWORD数组中的元素。

除此之外还要判断当前符号是否为’-‘号,如果是的话,需要使负号的个数加一,并
提前退出循环。

\begin{lstlisting}
    str2RevNum proc stdcall USES edx ecx eax 
                    string:ptr byte, 
                    numbers:ptr dword, 
                    len: dword
    xor ah, ah
    mov edx, string
    mov ebx, numbers
    ; i
    mov ecx, len
    sub ecx, 1
    jmp L3
L2:
    ; body
    mov al, [edx][ecx]

    .if al == 45
        add negNum, 1
        ret
    .endif  

    ; - '0'
    sub al, 30H

    ; 补0
    mov [ebx], eax
    add ebx, 4

    sub ecx, 1
L3:
    ; 循环终止条件判断
    cmp ecx, 0
    jns L2
    ;invoke      strlen,     offset   

    ret
str2RevNum endp
\end{lstlisting}

\subsection{乘法}
\begin{algorithm}[H]
    \SetAlgoLined
    \KwIn{两个数组的长度和首地址}
    \KwResult{结果数组的首地址}
    $i \leftarrow 0$ \;
    $j \leftarrow 0$ \;
    \For{i < len1}{
        \For{j < len2} {
            ans[i + j] += a[i] * b[j]
        }
    }
    \caption{大数乘法}
\end{algorithm}

即使用二重循环遍历两个数组,将他们的乘积加在结果数组中下标为i + j的元素
中。因为这里使用的是DWORD储存的乘积的和,而DWORD的最大范围是(1 << 32 - 1)
所以基本不用考虑在运算中DWORD会溢出,故可以把进位操作留在最后完成,这样可以很大
程度上加快程序的执行时间。具体代码如下:

\begin{lstlisting}
    multipy proc stdcall USES edi esi ecx edx ebx 
                        nums1:ptr dword, len1:dword, 
                        nums2:ptr dword, len2:dword, 
                        res:ptr dword
    mov edx, nums1
    mov ebx, nums2
    mov eax, res

    mov esi, 0
    jmp L2
L5:
    ; 第一层body
    mov edi, 0
    jmp L3
L4:
    ; 第二层body
    mov ecx, [edx + esi * 4]
    imul ecx, [ebx + edi * 4] ; a[i] * b[j]
   
    ; 计算下标
    add edi, esi

    ; res[i + j] += a[i] * b[j]
    add [eax + edi * 4], ecx 

    ; 还原edi
    sub edi, esi

    inc edi
L3:
    ; 第二层判断
    cmp edi, len2
    jl L4

    inc esi
L2:
    ; 第一层判断
    cmp esi, len1
    jl  L5

    ret
multipy endp
\end{lstlisting}


\subsection{进位}
\begin{algorithm}[H]
    \SetAlgoLined
    \KwIn{结果数组的首地址}
    \KwResult{无}
    $i \leftarrow 0$ \;
    \For{i < len}{
        \If {a[i] >= 10} {
            a[i + 1] = a[i] / 10 \;
            a[i] = a[i] \% 10 \; 
        }
    }
    \caption{大数乘法}
\end{algorithm}

即顺序扫描结果数组,如果该元素大于10则对10做除法,商进位到下一位,余数作为这一位
的数据。具体代码如下:

\begin{lstlisting}
    carry proc stdcall USES ebx edi edx res:ptr dword
    local beforeLen, afterLen

    ; 进位之前的长度
    invoke      getResLen,   res
    mov beforeLen, eax

    mov ebx, res
    mov edi, 0
L3:
    cmp edi, beforeLen
    jnl  L2
    
    mov eax, [ebx + edi * 4]
    xor edx, edx
    mov ecx, 10
    div ecx
    ; eax 结果 , edx 余数
    mov [ebx + edi * 4], edx
    add [ebx + edi * 4 + 4], eax

    inc edi
    jmp L3
L2:
    ; 进位之后长度
    invoke      getResLen,   res
    ret
carry endp
\end{lstlisting}


\section{实验结果展示}
输入多则测试用例,并与Python的计算结果相比较。

\subsubsection{正数与正数相乘}
输入两个9999999999999999999999999999999999999,并求乘积。

\begin{figure}[H]
    \centering
    \includegraphics[width= 0.9\textwidth]{e:/CourseProject/Assembly/bigNumberMul/assets/2020-06-04-20-11-20.png}
    \caption{验证1}
    \label{验证1}
\end{figure}

输入任意两个正数
\begin{figure}[H]
    \centering
    \includegraphics[width= 0.9\textwidth]{e:/CourseProject/Assembly/bigNumberMul/assets/2020-06-04-20-16-08.png}
    \caption{验证2}
    \label{验证2}
\end{figure}

\subsubsection{正数与负数相乘}

输入任意一个正数和负数。

\begin{figure}[H]
    \centering
    \includegraphics[width= 0.9\textwidth]{e:/CourseProject/Assembly/bigNumberMul/assets/2020-06-04-20-22-34.png}
    \caption{验证3}
    \label{验证3}
\end{figure}


\subsubsection{负数与负数相乘}

输入任意两个负数。

\begin{figure}[H]
    \centering
    \includegraphics[width= 0.9\textwidth]{e:/CourseProject/Assembly/bigNumberMul/assets/2020-06-04-20-24-16.png}
    \caption{验证4}
    \label{验证4}
\end{figure}

