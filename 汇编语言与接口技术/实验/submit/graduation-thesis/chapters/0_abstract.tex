%%
% The BIThesis Template for Bachelor Graduation Thesis
%
% 北京理工大学毕业设计(论文)中英文摘要 —— 使用 XeLaTeX 编译
%
% Copyright 2020 Spencer Woo
%
% This work may be distributed and/or modified under the
% conditions of the LaTeX Project Public License, either version 1.3
% of this license or (at your option) any later version.
% The latest version of this license is in
%   http://www.latex-project.org/lppl.txt
% and version 1.3 or later is part of all distributions of LaTeX
% version 2005/12/01 or later.
%
% This work has the LPPL maintenance status `maintained'.
%
% The Current Maintainer of this work is Spencer Woo.

% 中英文摘要章节
\zihao{-4}
\vspace*{-11mm}

\begin{center}
  \heiti\zihao{-2}\textbf{\thesisTitle}
\end{center}

\vspace*{2mm}

{\let\clearpage\relax \chapter*{\textmd{摘~~~~要}}}
\addcontentsline{toc}{chapter}{摘~~~~要}
\setcounter{page}{1}

\vspace*{1mm}

\setstretch{1.53}
\setlength{\parskip}{0em}

% 中文摘要正文从这里开始
本文使用MASM,以visual studio 2019作为开发的IDE,选取并实现了所给五个实验中的三个,分别为实验二——大数相乘;实验四——Windows界面风格实现两个文本文件内容的比对;实验五——汇编中多重循环的分析与实现。

从中掌握了基本的汇编语言编程技巧,同时也加强了对C语言编程与优化的理解。

% \textcolor{blue}{摘要正文选用模板中的样式所定义的“正文”,每段落首行缩进 2 个字符;或者手动设置成每段落首行缩进 2 个汉字,字体:宋体,字号:小四,行距:固定值 22 磅,间距:段前、段后均为 0 行。阅后删除此段。}

% \textcolor{blue}{摘要是一篇具有独立性和完整性的短文,应概括而扼要地反映出本论文的主要内容。包括研究目的、研究方法、研究结果和结论等,特别要突出研究结果和结论。中文摘要力求语言精炼准确,本科生毕业设计(论文)摘要建议 300-500 字。摘要中不可出现参考文献、图、表、化学结构式、非公知公用的符号和术语。英文摘要与中文摘要的内容应一致。阅后删除此段。}

\vspace{4ex}\noindent\textbf{\heiti 关键词:汇编语言与接口技术;MASM;大数相乘;文件对比;汇编多重循环}
\newpage

% 英文摘要章节
\vspace*{-2mm}

\begin{spacing}{0.95}
  \centering
  \heiti\zihao{3}\textbf{\thesisTitleEN}
\end{spacing}

\vspace*{17mm}

{\let\clearpage\relax \chapter*{
  \zihao{-3}\textmd{Abstract}\vskip -3bp}}
\addcontentsline{toc}{chapter}{Abstract}
\setcounter{page}{2}

\setstretch{1.53}
\setlength{\parskip}{0em}

% 英文摘要正文从这里开始

This article uses MASM, using visual studio 2019 as the development IDE, selects and implements three of the five experiments given, which are experiment two(multiplication of large numbers); experiment four(Windows interface style to achieve two text files content comparison); Experiment 5(Analysis and implementation of multiple cycles in compilation).

Finally mastered the basic assembly language programming skills, and also strengthened the understanding of C language programming and optimization.

% \textcolor{blue}{Abstract 正文设置成每段落首行缩进 2 字符,字体:Times New Roman,字号:小四,行距:固定值 22 磅,间距:段前、段后均为 0 行。阅后删除此段。}

\vspace{3ex}\noindent\textbf{Key Words: 
Assembly language and interface technology; MASM; Large number multiple; Files content comparison; Multiple cycles}
\newpage
