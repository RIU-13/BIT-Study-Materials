\chapter{实验五——C语言多重循环反汇编分析}

\section{实验目的}
C语言编写多重循环程序(大于3重),查看其反汇编码,分析各条语句功能(分析情况需要写入实验报告),并采用汇编语言重写相同功能程序。

\section{实验步骤}
\begin{enumerate}
    \item 编写C语言四重循环的代码
    \item 在Visual Studio 2019中查看其反汇编代码并调试
    \item 分析反汇编代码各条语句的功能
    \item 采用MASM编写具有同样功能的代码
\end{enumerate}

\section{多重循环的C语言源码}
该程序作用为,从一个数组中选取四个数(可以相同),并计算这四个数
的和,输出这个数组中所有可能的和。

\begin{lstlisting}[language=C]
#include <stdio.h>

int a[10] = { 0, 1, 2, 3, 4, 5, 6, 7, 8 ,9 };

int main() {
    int n = 10;
    for (int i = 0; i < n; i++) {
        for (int j = 0; j < n; j++) {
            for (int k = 0; k < n; k++) {
                for (int m = 0; m < n; m++) {
                    int tmp = a[i] + a[j] + a[k] + a[m];
                    printf("%d\n", tmp);
                }
            }
        }
    }
}
\end{lstlisting}

\section{反汇编代码分析}
采用Visual Studio 2019自带的反汇编查看功能,对该程序生成的
反汇编码进行查看并分析各条语句的功能。

\subsection{初始化}

具体汇编码如下:
\begin{lstlisting}
005F4E90  push        ebp  
005F4E91  mov         ebp,esp  
005F4E93  sub         esp,108h  
005F4E99  push        ebx  
005F4E9A  push        esi  
005F4E9B  push        edi  
005F4E9C  lea         edi,[ebp-108h]  
005F4EA2  mov         ecx,42h  
005F4EA7  mov         eax,0CCCCCCCCh  
005F4EAC  rep stos    dword ptr es:[edi]  
005F4EAE  mov         ecx,offset _7E81AEF5_main@c (05FC003h)  
005F4EB3  call        @__CheckForDebuggerJustMyCode@4 (05F1208h)  
\end{lstlisting}

对上述汇编代码进行分析,进入main函数之后主要的初始化的流程如下:
\begin{enumerate}
    \item push        ebp: 保存调用者的栈指针
    \item mov         ebp,esp:使栈指针与帧指针的值相同
    \item sub         esp,108h:为当前函数开辟局部变量的空间, 这一步同上一步共同建立了一个新的栈帧
    \item line4~line6: 因为C语言函数中不允许修改这三个寄存器的值,因此把他们保存在堆中
    \item lea         edi,[ebp-108h]: 把栈帧的最低地址复制给edi
    \item line8~line10: 对栈帧中未初始化的数据进行初始化,其中ecx的值为DWORD的数量,eax的值为要储存的值,0CCCCCCCCh即代表未被初始化的值。rep指令的目的是重复其上面的指令,ECX的值是重复的次数.
\end{enumerate}

\subsection{定义局部变量并赋值}

具体汇编码如下:
\begin{lstlisting}
    ;int n = 10;
005F4EB8  mov         dword ptr [n],0Ah  
\end{lstlisting}

其中[n]代表着直接寻址。

\subsection{循环体}

按照编译原理中的基本块划分的思想,可以把这段代码做如下划分:
\begin{lstlisting}

    ;for (int i = 0; i < n; i++) {
005F4EBF  mov         dword ptr [ebp-14h],0     <——
005F4EC6  jmp         main+41h (05F4ED1h)  

\end{lstlisting}

\begin{lstlisting}

005F4EC8  mov         eax,dword ptr [ebp-14h]   <——
005F4ECB  add         eax,1  
005F4ECE  mov         dword ptr [ebp-14h],eax  

\end{lstlisting}

\begin{lstlisting}
    005F4ED1  mov         eax,dword ptr [ebp-14h]   <——
    005F4ED4  cmp         eax,dword ptr [n]  
    005F4ED7  jge         main+0E5h (05F4F75h)  
            ;for (int j = 0; j < n; j++) {
\end{lstlisting}

\begin{lstlisting}
    005F4EDD  mov         dword ptr [ebp-20h],0     <——
    005F4EE4  jmp         main+5Fh (05F4EEFh)  
\end{lstlisting}

\begin{lstlisting}
    005F4EE6  mov         eax,dword ptr [ebp-20h]   <——
    005F4EE9  add         eax,1  
    005F4EEC  mov         dword ptr [ebp-20h],eax 
\end{lstlisting}

\begin{lstlisting}
    005F4EEF  mov         eax,dword ptr [ebp-20h]   <——
    005F4EF2  cmp         eax,dword ptr [n]  
    005F4EF5  jge         main+0E0h (05F4F70h)  
                ;for (int k = 0; k < n; k++) {
\end{lstlisting}

\begin{lstlisting}
    005F4EF7  mov         dword ptr [ebp-2Ch],0     <——
                ;for (int k = 0; k < n; k++) {
    005F4EFE  jmp         main+79h (05F4F09h)  
\end{lstlisting}

\begin{lstlisting}
    005F4F00  mov         eax,dword ptr [ebp-2Ch]   <——
    005F4F03  add         eax,1  
    005F4F06  mov         dword ptr [ebp-2Ch],eax  
\end{lstlisting}

\begin{lstlisting}
    005F4F09  mov         eax,dword ptr [ebp-2Ch]   <——
    005F4F0C  cmp         eax,dword ptr [n]  
    005F4F0F  jge         main+0DBh (05F4F6Bh)  
                    ;for (int m = 0; m < n; m++) {
\end{lstlisting}

\begin{lstlisting}
    005F4F11  mov         dword ptr [ebp-38h],0     <——
    005F4F18  jmp         main+93h (05F4F23h)  
\end{lstlisting}


\begin{lstlisting}
    005F4F1A  mov         eax,dword ptr [ebp-38h]   <——
    005F4F1D  add         eax,1  
    005F4F20  mov         dword ptr [ebp-38h],eax  
\end{lstlisting}


\begin{lstlisting}
    005F4F23  mov         eax,dword ptr [ebp-38h]   <——
    005F4F26  cmp         eax,dword ptr [n]  
    005F4F29  jge         main+0D9h (05F4F69h)  
                        ;int tmp = a[i] + a[j] + a[k] + a[m];
\end{lstlisting}


\begin{lstlisting}
    005F4F2B  mov         eax,dword ptr [ebp-14h]   <——
                        ;int tmp = a[i] + a[j] + a[k] + a[m];
    005F4F2E  mov         ecx,dword ptr a (05FA000h)[eax*4]  
    005F4F35  mov         edx,dword ptr [ebp-20h]  
    005F4F38  add         ecx,dword ptr a (05FA000h)[edx*4]  
    005F4F3F  mov         eax,dword ptr [ebp-2Ch]  
    005F4F42  add         ecx,dword ptr a (05FA000h)[eax*4]  
    005F4F49  mov         edx,dword ptr [ebp-38h]  
    005F4F4C  add         ecx,dword ptr a (05FA000h)[edx*4]  
    005F4F53  mov         dword ptr [ebp-44h],ecx  
                        ;printf("%d\n", tmp);
    005F4F56  mov         eax,dword ptr [ebp-44h]  
    005F4F59  push        eax  
    005F4F5A  push        offset string "%d\n" (05F7BCCh)  
    005F4F5F  call        _printf (05F1375h)  
    005F4F64  add         esp,8  
                    ;}
    005F4F67  jmp         main+8Ah (05F4F1Ah)  
                ;}
\end{lstlisting}


\begin{lstlisting}
    005F4F69  jmp         main+70h (05F4F00h)     <——
            ;}
\end{lstlisting}


\begin{lstlisting}
    005F4F6B  jmp         main+56h (05F4EE6h)     <——
        ;}
\end{lstlisting}


\begin{lstlisting}
    005F4F70  jmp         main+38h (05F4EC8h)     <——
    ;}
\end{lstlisting}


\begin{lstlisting}
    005F4F75  xor         eax,eax                 <——
    005F4F77  pop         edi  
    005F4F78  pop         esi  
    005F4F79  pop         ebx  
    005F4F7A  add         esp,108h  
    005F4F80  cmp         ebp,esp  
    005F4F82  call        __RTC_CheckEsp (05F1212h)  
    005F4F87  mov         esp,ebp  
    005F4F89  pop         ebp  
    005F4F8A  ret
\end{lstlisting}

根据以上基本块的划分,可以画出程序的流程图如下:

\begin{figure}[H]
    \centering
    \includegraphics[width= 0.8\textwidth]{e:/CourseProject/Assembly/bigNumberMul/assets/tmp.pdf}
    \caption{循环体流程图}
    \label{循环体流程图}
\end{figure}

\section{采用MASM32重写该程序}

根据上述的程序流程图,在汇编中定义与其对应的代码块和跳转即可重写该
程序。与反汇编的代码的差别如下:
\begin{enumerate}
    \item 使用eax储存n,避免每次比较都需要执行mov eax, n
\end{enumerate}

部分关键代码如下:

\begin{lstlisting}
main proc    
    LOCAL i, j, k, m, tmp:DWORD
    mov eax, n

L0:
    mov i, 0
    jmp L2

L1:
    inc i

L2:
    cmp i, eax
    jge    L13    

L3:
    mov j, 0
    jmp L5

L4:
    inc j

L5:
    cmp j, eax
    jge L1

L6:
    mov k, 0
    jmp L8

L7:
    inc k

L8:
    cmp k, eax
    jge L4

L9:
    mov m, 0
    jmp L11

L10:
    inc m

L11:
    cmp m, eax
    jge    L7

L12:
    mov esi, i
    mov ecx, numbers[esi * 4]
    
    mov esi, j
    add ecx, numbers[esi * 4]

    mov esi, k
    add ecx, numbers[esi * 4]
    
    mov esi, m
    add ecx, numbers[esi * 4]

    mov tmp, ecx

    push eax
    invoke    printf, offset    string_d, tmp
    pop eax

    jmp L10

L13:
    ret    
main endp    
\end{lstlisting}

\section{实验结果展示}

由于四次循环会产生$n^4$个输出,因此本次测试让数组的长度为4,即
numbers = $\left\{1, 2, 3, 4\right\}$, 运行结果如下:

\begin{figure}[H]
    \centering
    \includegraphics[width= 0.9\textwidth]{e:/CourseProject/Assembly/bigNumberMul/assets/2020-06-05-15-11-24.png}
    \caption{多重循环实验结果-1}
    \label{多重循环实验结果-1}
\end{figure}

\begin{figure}[H]
    \centering
    \includegraphics[width= 0.9\textwidth]{e:/CourseProject/Assembly/bigNumberMul/assets/2020-06-05-15-12-04.png}
    \caption{多重循环实验结果-2}
    \label{多重循环实验结果-2}
\end{figure}

共64个输出,最小的为1+1+1+1=4,最大的为4+4+4+4=16,与C语言程序的
运行结果一致。
