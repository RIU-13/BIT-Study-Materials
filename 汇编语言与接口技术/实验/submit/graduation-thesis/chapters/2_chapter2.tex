\chapter{实验四——Windows界面风格的文本文件内容比对}

\section{实验目的}
实现一个windows界面风格的文本文件内容比较程序,如果两文件内容一样,输出相应提示;若两文件不一样,输出对应的行号。

\section{实验设计}

\subsection{GUI设计}
本次实验设计的GUI如图所示:
\begin{figure}[H]
    \centering
    \includegraphics[width= 0.9\textwidth]{e:/CourseProject/Assembly/bigNumberMul/assets/2020-06-04-23-43-45.png}
    \caption{GUI样式}
    \label{GUI样式}
\end{figure}


\subsection{设计思路}
创建一个主窗口,之后进入无限消息循环处理模式,当有点击消息产生的时候,根据eax
来判断点击的是哪个按钮,并做出相应的动作。

比较文件的时候,先打开两个文件,并把他们的句柄保存,之后逐行读取文件并进行比较,
最后弹出对话框显示文件的比较结果。

\subsection{程序流程}
\begin{enumerate}
    \item 点击文件1选择按钮,弹出文件选择对话框
    \item 选择文件1,记录文件1路径
    \item 点击文件2选择按钮,弹出文件选择对话框
    \item 选择文件2,记录文件2路径
    \item 点击比较按钮
    \item 分别从两个文件中读取两行
          \iitem 如果两行长度均为0,跳转至7
          \iitem 若有一行长度不为0,则进行比较,比较完成后跳转至6
    \item 将结果数组中的数字转化为字符串
    \item 弹出包含有比较结果的窗口
\end{enumerate}


\section{具体实现}
该节介绍了代码中的部分核心算法和具体代码。

\subsection{主窗口的创建过程}
主窗口的创建流程如下:
\begin{enumerate}
    \item 得到应用程序的句柄
    \item 初始化stWndClass
    \item 指定窗口过程
    \item 注册窗口
    \item 建立窗口
    \item 刷新窗口客户区
    \item 进入消息获取和处理的循环
\end{enumerate}

具体代码如下:
\begin{lstlisting}
_WinMain proc  
    local stWndClass:WNDCLASSEX 
    local stMsg:MSG
    invoke GetModuleHandle, NULL  ;得到应用程序的句柄
    mov hInstance, eax
    invoke RtlZeroMemory, addr stWndClass, \
                          sizeof stWndClass

    invoke LoadCursor, 0, IDC_ARROW
    mov stWndClass.hCursor, eax
    push hInstance
    pop stWndClass.hInstance
    mov stWndClass.cbSize, sizeof WNDCLASSEX
    mov stWndClass.style, CS_HREDRAW or CS_VREDRAW
    ; 指定窗口过程
    mov stWndClass.lpfnWndProc, offset _ProcWinMain
    mov stWndClass.hbrBackground, COLOR_WINDOW+1
    mov stWndClass.lpszClassName, offset szClassName
    ; 注册窗口
    invoke RegisterClassEx, addr stWndClass

    invoke CreateWindowEx, WS_EX_CLIENTEDGE,\
                           offset szClassName, \
                           offset szCaptionMain,\  
                           WS_OVERLAPPEDWINDOW, \
                           100, 100, 500, 200,\
                           NULL, NULL, hInstance, NULL
    mov hWinMain, eax  ;句柄放入hWinMain
    invoke ShowWindow, hWinMain, SW_SHOWNORMAL  
    ;刷新窗口客户区
    invoke UpdateWindow, hWinMain  

    .while TRUE  ;消息获取和处理的循环
        invoke GetMessage, addr stMsg, NULL, 0, 0
        .break .if eax==0  ;退出消息
        invoke TranslateMessage, addr stMsg  
        invoke DispatchMessage, addr stMsg
    .endw
    ret
_WinMain endp

\end{lstlisting}

\subsection{窗口过程函数}

窗口过程函数指定了窗口针对不同事件的处理方式。本次实验中事件的类型如下:

\begin{enumerate}
    \item WM\_CREATE:窗口创建信号, 绘制窗口的按钮以及文本框组件
    \item WM\_COMMAND:按钮点击信号,根据不同的句柄来调用不同的处理函数
        \iitem eax == 1: 弹出文件浏览窗口,选择后改变textEdit1中的内容
        \iitem eax == 3: 弹出文件浏览窗口,选择后改变textEdit2中的内容   
        \iitem eax == 5: 调用比较函数,并弹出新的对话框展示比较的内容   
    \item WM\_CLOSE:窗口关闭信号,销毁窗口
\end{enumerate}


\subsection{文件浏览并选择}
本次实验中,通过调用Windows的系统API中的SHBrowseForFolder函数来实现浏览文件
并选择文件。

调用流程如下:
\begin{enumerate}
    \item 创建BROWSEINFO结构体并初始化
    \item 设置BROWSEINFO\.ulFlags为BIF\_BROWSEINCLUDEFILES,使其可以显示文件
    \item 调用SHBrowseForFolder
    \item 判断返回值eax的大小
        \iitem eax == 0, 说明未选择文件,直接返回
        \iitem eax != 0, 说明选择了文件,跳转至5
    \item 调用SHGetPathFromIDList得到完整的文件名
    \item 调用CoTaskMemFree释放空间
\end{enumerate}

\subsection{读取一行}

通过传入的句柄,从文件中读取一行到缓冲区中,流程如下:
\begin{enumerate}
    \item 调用ReadFile函数,并设置每次只读取一个字符
    \item 判断读取的字符长度是否为0
        \iitem 若为0,跳转至4
        \iitem 若不是,将该字符添加到缓冲区
    \item 判断该字符是否为换行符
        \iitem 若是换行符,跳转至4
        \iitem 若不是,跳转至1
    \item 在缓冲区中添加'0'作为字符串结束符
    \item 调用lstrlen得到缓冲区字符串长度
\end{enumerate}

具体代码如下:
\begin{lstlisting}
ReadLine proc stdcall uses ebx, 
                      fileHand: HANDLE, 
                      lpLineBuf: ptr byte
    ; 读完文件 eax为0
    LOCAL br     :DWORD
    local char   :BYTE
    mov ebx, lpLineBuf

L2:
    invoke ReadFile,fileHand,addr char,1,ADDR br,NULL
    cmp     br, 0
    je      L1

    mov     al, char  
    mov     [ebx], al
    inc     ebx
    
    cmp     char, 10
    je      L1
    
    jmp     L2
L1:
    mov     al, 0
    mov     [ebx], al

    invoke  lstrlen, lpLineBuf
    ret     
ReadLine endp

\end{lstlisting}

\subsection{文件比较}

通过传入的两个文件路径,对两个文件进行比较,当内容不同时,把当前的
行号记录在数组中,流程如下:

\begin{enumerate}
    \item 调用CreateFile来得到两个文件的句柄
    \item 从两个文件中分别读取一行
        \iitem 两行的长度都是0,则退出
        \iitem 如果有一行的长度不为0
            \iiitem 调用lstrcmp进行比较
            \iiitem 若结果不同,将当前行数记录在结果数组中
            \iiitem 跳转至2
\end{enumerate}

具体代码如下:
\begin{lstlisting}
Compare proc stdcall uses ebx eax edx 
                     lpFilePath1: ptr byte, 
                     lpFilePath2: ptr byte
    LOCAL br     :DWORD
    LOCAL len    :DWORD
    LOCAL lineLen1, lineLen2:DWORD
    LOCAL lineNum:DWORD

    mov   eax, 0
    mov   lineNum, eax

    mov   ebx, offset diffLines
    invoke CreateFile, lpFilePath1,
                    GENERIC_READ,
                    FILE_SHARE_READ,
                    NULL,OPEN_EXISTING,
                    FILE_ATTRIBUTE_NORMAL,
                    NULL
    mov fileHandle1, eax
    
    invoke CreateFile, lpFilePath2,
                GENERIC_READ,
                FILE_SHARE_READ,
                NULL,OPEN_EXISTING,
                FILE_ATTRIBUTE_NORMAL,
                NULL
    mov fileHandle2, eax

    .WHILE  TRUE
        inc lineNum

        invoke  ReadLine, fileHandle1,\
                          offset lineBuf1
        mov     lineLen1, eax

        invoke  ReadLine, fileHandle2,\
                          offset lineBuf2
        mov     lineLen2, eax

        ; 两个line长度都是0就退出
        add     eax, lineLen1
        .BREAK  .IF  eax == 0

        invoke  lstrcmp, offset lineBuf1,\
                         offset lineBuf2

        .IF eax != 0
            mov     eax, lineNum
            mov     [ebx], eax
            add     ebx, 4

            inc     diffNum
        .ENDIF
    .ENDW

    ret
Compare endp
\end{lstlisting}

\section{实验结果展示}

\subsubsection{比较两个相同的文件}

1. 选取文件test.txt作为文件1
\begin{figure}[H]
    \centering
    \includegraphics[width= 0.9\textwidth]{e:/CourseProject/Assembly/bigNumberMul/assets/2020-06-04-23-46-18.png}
    \caption{选择文件test.txt作为文件1}
    \label{选择文件test.txt作为文件1}
\end{figure}


2. 选取文件test.txt作为文件2
\begin{figure}[H]
    \centering
    \includegraphics[width= 0.9\textwidth]{e:/CourseProject/Assembly/bigNumberMul/assets/2020-06-04-23-47-32.png}
    \caption{选择文件test.txt作为文件2}
    \label{选择文件test.txt作为文件2}
\end{figure}

3. 比较
\begin{figure}[H]
    \centering
    \includegraphics[width= 0.9\textwidth]{e:/CourseProject/Assembly/bigNumberMul/assets/2020-06-04-23-48-36.png}
    \caption{显示文件相同}
    \label{显示文件相同}
\end{figure}

\subsubsection{比较两个不同的文件}

分别选取test.txt, test1.txt作为文件1和文件2,并进行比较,选择完之后如图:
\begin{figure}[H]
    \centering
    \includegraphics[width= 0.9\textwidth]{e:/CourseProject/Assembly/bigNumberMul/assets/2020-06-04-23-51-04.png}
    \caption{选择两个不同的文件}
    \label{选择两个不同的文件}
\end{figure}

两个文件的内容分别如下,不同的地方已用红线标出:
\begin{figure}[H]
    \centering
    \includegraphics[width= 0.9\textwidth]{e:/CourseProject/Assembly/bigNumberMul/assets/2020-06-04-23-54-34.png}
    \caption{两个文件的内容}
    \label{两个文件的内容}
\end{figure}


比较结果如下:
\begin{figure}[H]
    \centering
    \includegraphics[width= 0.9\textwidth]{e:/CourseProject/Assembly/bigNumberMul/assets/2020-06-04-23-55-54.png}
    \caption{不同的行数}
    \label{不同的行数}
\end{figure}

